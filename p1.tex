\documentclass[letterpaper,10pt,draftclsnofoot,onecolumn]{IEEEtran}
\usepackage{graphicx}                                        
\usepackage{amssymb}                                         
\usepackage{amsmath}                                         
\usepackage{amsthm}                                          

\usepackage{alltt}                                           
\usepackage{float}
\usepackage{color}
\usepackage{url}

\usepackage{balance}
\usepackage[TABBOTCAP, tight]{subfigure}
\usepackage{enumitem}
\usepackage{pstricks, pst-node}

\usepackage[margin=0.75in]{geometry}
\geometry{textheight=8.5in, textwidth=6in}

%random comment

\newcommand{\cred}[1]{{\color{red}#1}}
\newcommand{\cblue}[1]{{\color{blue}#1}}

\newcommand{\toc}{\tableofcontents}

%\usepackage{hyperref}

\def\name{Jiaxu Li, Xiaoyi Yang}

%pull in the necessary preamble matter for pygments output
\input{pygments.tex}

%% The following metadata will show up in the PDF properties
\hypersetup{
   colorlinks = true,
   urlcolor = black,
   pdfauthor = {\name},
   pdfkeywords = {cs444 ``operating systems 2'' Project 1},
   pdftitle = {CS 444 Project 1: Getting Acquainted},
   pdfsubject = {CS 444 Project 1},
   pdfpagemode = UseNone
  }

\parindent = 0.0 in
\parskip = 0.1 in

\begin{document}
\begin{titlepage}
		
		\begin{center}
		\bigbreak	
		\textbf{Project 1: Getting Acquainted}
		by Jiaxu Li and Xiaoyi Yang from Group 34
		CS 444 Operating System 2 Fall 2017
		\end{center}
		
		Abstract: This report is to list what commands we used to build kernel, explain each and every flag in the listed qemu command-line, and provide sufficient detail for the concurrency version control log and work log. 
	\end{titlepage}
%\tableofcontents

%input the pygmentized output of mt19937ar.c, using a (hopefully) unique name
%this file only exists at compile time. Feel free to change that.


\section*{Assignment 1}
\textbf{Due: 23:59:59, 9 October 2017}
\subsection*{A log of commands used to perform the requested actions}
\begin{itemize}
\item 	ssh os2
\item 	cd /scratch/fall2017
\item 	mkdir 34
\item 	cd 34
\item 	git clone  git://git.yoctoproject.org/linux-yocto-3.19
\item 	git checkout v3.19.2  
\item 	source /scratch/opt/environment-setup-i586-linux
\item 	cp /scratch/files/config-3.19.2-yocto-qemu .config
\item 	Make menuconfig (change local version)
\item 	Make -j4 all
\item 	qemu-system-i386 -gdb tcp::5534 -S -nographic -kernel bzImage-qemux86.bin -drive file=core-image-lsb-sdk-qemux86.ext4,if=virtio -enable-kvm -net none -usb -localtime --no-reboot --append "root=/dev/vda rw console=ttyS0 debug".
\item 	$GDB linux-yocto-$3.19$/vmlinux
\item 	continue
\item 	target remote:$5534$
\item 	root
\item 	uname -a
\item 	reboot
\item 	qemu-system-i$386$ -gdb tcp::$5534$ -S -nographic -kernel linux-yocot$3.19$/arch/x$86$/boot/bzImage -drive file=core-image-lsb-sdk-qemux$86$.ext$4$,if=virtio -enable-kvm -net none -usb -localtime --no-reboot --append "root=/dev/vda rw console=ttyS$0$ debug".
\item 	q
\end{itemize}



\subsubsection*{An explanation of each and every flag in the listed qemu } 
\begin{itemize}
	\item The code: qemu-system-i$386 -gdb tcp::5534 -S -nographic -kernel bzImage-qemux86.bin -drive file=core-image-lsb-sdk-qemux86.ext4,if=virtio -enable-kvm -net none -usb -localtime --no-reboot --append "root=/dev/vda rw console=ttyS0 debug".
	\item -gdb tcp::5534 : open a gdbserver on TCP port 5534 
	\item -S: do not start CPU at startup
	\item -nongraphic: disable graphical output so that QEMU is a simple command line application
	\item -kernel bzImage-qemux86.bin: use bzImage as kernel image
	\item -drive: define which disk image to use with this drive
	\item -if=virtio: specify the controller's PCI
	\item -enable-kvm: enable KVM full virtualization support
	\item -net none: indicate that no network devices should be configured
	\item -usb: enable USB drivers
	\item -localtime: let the RTC start at the current UTC or local time
	\item --no-reboot: exit instead of rebooting
	\item --append: use cmdline as kernel command line
\end{itemize}
\subsubsection*{Write-up of Concurreny solution}
Thhe concurrency problem was Implementated in C language. After reading assignment material, we know that there are several synchronization constraints that we need to achieve. We keep the buffer in an inconsistent state while threads are working with buffer. Also, the consumer threads would be blocked if the buffer is empty and the producer threads would be blocked if the buffer is empty. We put the iems in the buffer in a struct. And the item are generated by Mersenne Twister.
\subsubsection*{Reflection}
\begin{itemize}
\item What do you think the main point of this assignment is?
       We thihnk the main point of this assignment is to make us think in parallel (multi-threaded programs) and improve our solving problems skill.
\item How did you personally approach the problem? Design decisions, algorithm, etc.
       We start to know about what The Producer-Consumer Problem is. Then, we searched information about multithreaded program to refresh our memory. And follow the material to slove problems piece by piece.   
\item How did you ensure your solution was correct? Testing details, for instance.
      We followed the material and would not proceed any further before we make sure our function worked correctly. For instance, the producer can add itmes in the buffer. And the items did exist in the buffer.
\item What did you learn?
	   We refreshed our knowledge about multithreaded program and C language.
\end{itemize}
\subsection*{Work Log}
2017/10/04       We started to read assignment and make sure what we need to do. And do some research.
2017/10/05       We started to build our kernal following the material. But the things were not going well. We tried to do some search online and asked help from our classmates. We thought we build our kernel sucessfully. And started to work on concurrency problems.
2017/10/06       Got much useful information from WebEx conference calls. And we wanted to build our kernel again to make sure we did correctly. And continuted to work on concurrency problems.
2017/10/07       Still worked on concurrency and did some debugs because we got some problems. And started to work on Write-Up.
2017/10/08       Finished debugs and kept working on Latex. 
%\input{__mt19937ar.c.tex}

\end{document}
